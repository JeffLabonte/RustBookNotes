\documentclass{article}
\usepackage{graphicx}
\usepackage{hyperref}

\begin{document}
  \begin{titlepage}
    \begin{center}
      \line(1,0){300} \\
      [0.25in]
      \huge{\bfseries Chapter 13 - Iteration and Closure}\\
      [2mm]
      \line(1,0){300}\\
      [1.5cm]
      \textsc{\LARGE Jeff labonte}\\
      [0.75cm]
    \end{center}
  \end{titlepage}
 \section {Closure}
 \ \\[2mm]
 Closure are used to create an anonymous function \\
 \ \\[2mm]
 It can be used to create a small function  that you can \\
 store in a variable and move around\\
 \ \\[2mm]
 It is possible to create a struct that takes in a function ( closure ) \\
 and a variable to store the information returned by the closure \\
 \ \\[2mm]
 \section{Iterator}
 \ \\[2mm]
 Iterators is a way to create an object that is built to go 
 through a vector or any other collections. They are known to be 
 lazy which means they do no effect until they are used.\\
 Iterators must be spended! \\
 \ \\[2mm]
 There are multiple implementations of iterations \\
 \begin{itemize}
     \item The consuming adaptor
         \begin{itemize}
             \item This one takes ownerships over the data in the iterator
         \end{itemize}
     \item The iterator adaptor
         \begin{itemize}
             \item Example of implementation : map
         \end{itemize}
 \end{itemize}
\end{document}
